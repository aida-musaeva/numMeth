\documentclass[12pt,a4paper]{article}
\usepackage[utf8]{inputenc}
\usepackage[russian]{babel}
\usepackage[width=17cm,top=1cm,height=27cm]{geometry}
\title{Численные методы(Мусаева)}

\date{May 2019}

\begin{document}
\begin{titlepage}
\begin{center}
2019 год
\vspace {8cm}



{ \LARGEПрактикум по численным методам:\\ вычисление функции}\\
\vspace {8cm}
\bigskip Мусаева Аида, группа 208
\end{center}
\vfill


\vfill

\end{titlepage}

\section{Цель работы}
\text{Целью работы является овладение практическим навыком решения обратной задачи теории погрешностей, а также применение результатов этой работы в вычислении значения функции.}
\section{Постановка задачи}
\text{
1) По указанной точности (\epsilon=10$^{-6}$)
решить обратную задачу теории погрешности для функции $z(x) = \sqrt{ sin(x+0.74)} sh(0.8x^{2}+0.1)$, где x=0.1(0.01)0.2.

2) Построить с требуемой точностью таблицу значений этой функции (квадратный корень вычислять по формуле Герона, остальные простейшие элементарные функции вычислять с использованием степенных рядов).

3) Составить ту же таблицу, используя встроенные функции и сравнить обе таблицы.
}

\section{Аналитические вычисления}
\text{$u(x)=sin(x+0.74)\\
v(x)=sh(0.8x^{2}+0.1)\\
f(u, v) = \sqrt{u}*v$\\
Найдем пределы изменения величин $u, v$ при $x \in [0.1;0.2]$. Функции $u(x), v(x)$ монотонно возрастают на [0.1;0.2]. Интервал изменения $u, v$ можно расширить, чтобы не вычислять верхние и нижние границы изменения этих функций с большой точностью. На промежутке [0.1; 0.2]: 
0.74 <= $u$ <= 0.81,
0.1 <= $v$ <= 0.14//
Таким образом, $G$ = \big\{$(u, v)$: 0.74 <= $u <= 0.81,
0.1 <= v <= 0.14\big\}\\
|f_u|=|\frac{v}{2\sqrt{u}}|<0.09\\
|f_v|=|\sqrt{u}|<0.9\\$
Таким образом, $\epsilon_u = \frac{10^{-6}}{0.27}, \epsilon_v = \frac{10^{-6}}{2.7}, \epsilon_f = \frac{10^{-6}}{3}
$}
\section{Код программы}
\begin{verbatim}
#include "stdafx.h"
#include <iostream>
#include <math.h>

using namespace std;

long int factorial(int i) {
	if (i == 0) return 1;
	else return i * factorial(i - 1);
}

long double sqr(long double n, const double eps) {
	long double x = 1, nx;
	while (true)
	{
		nx = (x + n / x) / 2;
		if (fabs(x - nx) < eps)
			break;
		else x = nx;
	}
	return x;
}

long double sh(double n, const double eps) {
	long double ans = 0;
	int k = 0;

	while (true)
	{
		long double u = pow(n, 2 * k + 1) / factorial(2 * k + 1);
		ans += u;
		if (abs(u) < eps)
			break;
		k++;
	}
	return ans;
}

long double sin(long double n, const double eps) {
	long double ans = 0;
	int k = 0;

	while (true)
	{
		long double u = pow(n, 2 * k + 1) / factorial(2 * k + 1);
		ans += pow(-1, k)*u;

		if (abs(u) < eps)
			break;
		k += 1;
	}
	return ans;
}

long double f(long double x) {
	return sqr(sin(x + 0.74, pow(10, -6) / 0.27), pow(10, -6) / 3)*sh(0.8*pow(x,2)+0.1, pow(10, -6) / 2.7);
}

long double math_f(long double x) {
	return sqrt(sin(x+0.74))*sinh(0.8*pow(x,2)+0.1);
}

int main() {

	for (double x = 0.1; x <= 0.2; x += 0.01) {
		double res1 = f(x);
		cout << res1 << endl;
	}

	cout << endl;

	for (double x = 0.1; x <= 0.2; x += 0.01)
	{
		double res2 = math_f(x);
		cout << res2 << endl;
	}

	cout << endl;

	for (double x = 0.1; x <= 0.2; x += 0.01) {

		double res3 = fabs(math_f(x) - f(x));
		cout << res3 << endl;
	}
}
\end{verbatim}
\section{Таблицы}
\text{Значения при использовании моей реализации функций}\\
\\
\begin{tabular}{|x|z|}
\hline
$x$ &$z(x)$ \\
0.1 & 0.09337740236\\
\hline
0.11 & 0.09525743088\\
\hline
0.12 & 0.09728414920\\
\hline
0.13 & 0.09945918858\\
\hline
0.14 & 0.10178413613\\
\hline
0.15 & 0.10426053623\\
\hline
0.16 & 0.10688989193\\
\hline
0.17 & 0.10967366641\\
\hline
0.18 & 0.11261328449\\
\hline
0.19 & 0.11571013413\\
\hline
\end{tabular}\\
\\
\\
\text{Значения при использовании стандартных функций}\\
\begin{tabular}{|x|z|}
\hline
$x$ &$z(x)$ \\
0.1 & 0.09337740200\\
\hline
0.11 & 0.09525743052\\
\hline
0.12 & 0.09728414883\\
\hline
0.13 & 0.09945918819\\
\hline
0.14 & 0.10178413571\\
\hline
0.15 & 0.10426053577\\
\hline
0.16 & 0.10688989142\\
\hline
0.17 & 0.10967366584\\
\hline
0.18 & 0.11261328385\\
\hline
0.19 & 0.11571013341\\
\hline
\end{tabular}

\vspace{10cm}
\text{Разница между значениями}\\
\\
\begin{tabular}{|x|}
\hline
$\Delta$ \\
\hline
3.60824e-10\\
\hline
3.59994e-10\\
\hline
3.6994e-10\\
\hline
3.9006e-10\\
\hline
4.20053e-10\\
\hline
4.59883e-10\\
\hline
5.09742e-10\\
\hline
5.7003e-10\\
\hline
6.41332e-10\\
\hline
7.24404e-10\\
\hline
\end{tabular}
\end{document}