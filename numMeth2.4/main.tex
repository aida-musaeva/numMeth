\documentclass[12pt,a4paper]{article}
\usepackage{graphics}
\usepackage{amssymb}
\usepackage{ragged2e}
\justifying
\usepackage[utf8]{inputenc}
\usepackage[russian]{babel}
\usepackage[width=17cm,top=1cm,height=27cm]{geometry}
\usepackage{graphicx}
\title{Численные методы(Мусаева)}

\date{November 2019}

\begin{document}
\begin{titlepage}
\begin{center}
2019 год
\vspace {8cm}

{ \LARGEПрактикум по численным методам: интерполяционный полином Эрмита }\\
\vspace {8cm}
\bigskip Мусаева Аида, группа 301
\end{center}
\vfill
\vfill
\end{titlepage}

\section{Теория}

\text{
Эрмитова интерполяция строит многочлен, значения которого в выбранных точках совпадают со значениями исходной функции в этих точках, и производные многочлена в данных точках совпадают со значениями производных функции (до некоторого порядка m). Это означает, что $n(m+1)$ величин}

\begin{matrix}

$\\
(x_0, y_0), &(x_1, y_1), &\ldots, &(x_{n-1}, y_{n-1}), \\
(x_0, y_0'), &(x_1, y_1'), &\ldots, &(x_{n-1}, y_{n-1}'), \\
\vdots & \vdots & &\vdots \\
(x_0, y_0^{(m)}), &(x_1, y_1^{(m)}), &\ldots, &(x_{n-1}, y_{n-1}^{(m)})
$
\end{matrix}

должны быть известны. Полученный многочлен может иметь степень не более, чем $n(m+1)−1$. (В общем случае $m$ не обязательно должно быть фиксировано, то есть в одних точках может быть известно значение большего количества производных, чем в других. В этом случае многочлен будет иметь степень $N−1$, где $N$ - число известных значений.)

Разделённой разностью нулевого порядка функции $f$ в точке $x_j$ называют значение $f(x_j)$, а разделённую разность порядка $k$ для системы точек $(x_j, \; x_{j+1}, \; \ldots, \; x_{j+k})$ определяют через разделённые разности порядка $(k-1)$ по формуле
: \\
$f(x_j; \; x_{j+1}; \; \ldots; \; x_{j+k-1}; \; x_{j+k}) = \frac{f(x_{j+1}; \; \ldots; \; x_{j+k-1}; \; x_{j+k}) - f(x_{j}; \; x_{j+1};\;\ldots;\;x_{j+k-1})}{x_{j+k}-x_{j}} ,$\\
в частности:\\
$f(x_0;\;x_1)=\frac{f(x_1)-f(x_0)}{x_1-x_0} ,$\\

$f(x_0;\;x_1;\;x_2)=\frac{f(x_1;\;x_2)-f(x_0;\;x_1)}{x_2-x_0}=\frac{\frac{f(x_2)-f(x_1)}{x_2-x_1}-\frac{f(x_1)-f(x_0)}{x_1-x_0}}{x_2-x_0} ,$ \\

$f(x_0;\;x_1;\;\ldots;\;x_{n-1};\;x_n) = \frac{f(x_1;\;\ldots;\;x_{n-1};\;x_n) - f(x_0;\;x_1;\;\ldots;\;x_{n-1})}{x_n-x_0} .$ \\

Для разделённой разности верна формула
: \\ $f(x_0;\;x_1;\;\ldots;\;x_n)=\sum_{j=0}^n\dfrac{f(x_j)}{\prod\limits_{i=0\atop i\neq j}^n(x_j-x_i)},$
в частности,
: $(x_0;\;x_1)=\frac{f(x_1)}{x_1-x_0}+\frac{f(x_0)}{x_0-x_1} ,$
: $f(x_0;\;x_1;\;x_2) = \frac{f(x_2)}{(x_2-x_1)(x_2-x_0)}+\frac{f(x_1)}{(x_1-x_2)(x_1-x_0)}+\frac{f(x_0)}{(x_0-x_2)(x_0-x_1)} .$\\

В общем случае полагаем, что в данных точках $x_i$ известны производные функции $f$ до порядка $k$ включительно. Тогда набор данных $z_0, z_1, \ldots, z_{N}$ содержит $k$ копий $x_i$. При создании таблицы разделенных разностей при $j = 2, 3, \ldots, k$ одинаковые значения будут вычислены как:\\

$\frac{f^{(j)}(x_i)}{j!}$.\\

Например
:\\$f[x_i, x_i, x_i]=\frac{f''(x_i)}{2}$ \\
$f[x_i, x_i, x_i, x_i]=\frac{f^{(3)}(x_i)}{6}$
и так далее.
\\
\\
Интерполяционный многочлен Эрмита получаем взятием коэффициентов диагонали таблицы разделенных разностей, и умножением коэффициента с номером $k$ на $\prod_{i=0}^{k-1} (x - z_i)$, как при получении многочлена Ньютона.
\section{Постановка задачи}
\begin{tabular}{xfff}
\hline
$x$ &$f(x)$ & $f'(x)$&$f''(x)$\\
\hline
1 & 13 & 10 & 23\\
\hline
2 & 3 & -3 & \\
\hline
5 & 2 & 2 & 130\\
\hline
\end{tabular}
\section{Код программы}
\begin{verbatim}
import numpy as np
import sympy
from sympy import *
x = Symbol('x')
X = np.array([1, 5, 2])
Y = np.array([13, 2, 3])
dY = np.array([10, 2, -3])
d2Y = np.array([23, 130])
omega = collect(expand((x-1) * (x-2) * (x-5)), x)
def HermitPolynomial(X,dY, d2Y):
    def pLagrange(X, Y):
        x = symbols('x')
        L = 0
        for j in range(np.prod(X.shape)):
            l = 1
            l_j = 1
            for i in range(np.prod(X.shape)):
                if i == j:
                    l *= 1
                    l_j *= 1
                else:
                    l *= (x - X[i])
                    l_j *= (X[j] - X[i])
            L += Y[j] * l / l_j
        return collect(expand(L), x)

    def P4():

        P4 = np.array([(dY[i] - pLagrange(X, Y).diff(x).subs(x, X[i])) 
            / omega.diff(x).subs(x, X[i]) for i in range(3)])
        dP4 = np.array([(d2Y[i] - pLagrange(X, Y).diff(x).diff(x).subs(x, X[i]) 
            - omega.diff(x).diff(x).subs(x, X[i]) *
                         P4[i]) / (2 * omega.diff(x)) for i in range(2)])

        def P1():
            P1 = np.array(
                [(dP4[i] - pLagrange(X, P4).diff(x).subs(x, X[i])) 
                    / omega.diff(x).subs(x, X[i]) for i in range(2)])
            coeff = Array([[-1255/13824 ], [-23755/13824]])
            P1 = coeff[0] * x + coeff[1]
            print(simplify(P1))
            return P1

        res = pLagrange(X, P4) + omega * P1()
        return res

    H = pLagrange(X, Y) + omega * P4()
    return simplify(H)

H = HermitPolynomial(X, dY, d2Y)
print(H)
\end{verbatim}
\section{Результат работы программы}
\begin{verbatim}
    -0.0907841435185185*x**7 - 0.265842013888889*x**6 + 20.4735243055556*x**5 
    - 169.662760416667*x**4 + 611.473741319444*x**3 - 1105.869140625*x**2 + 
    965.831018518518*x - 308.889756944444
\end{verbatim}
\end{document}


